% -------------------------------------------------------
% Daten für die Arbeit
% Wenn hier alles korrekt eingetragen wurde, wird das Titelblatt
% automatisch generiert. D.h. die Datei titelblatt.tex muss nicht mehr
% angepasst werden.

\newcommand{\hsmasprache}{de} % de oder en für Deutsch oder Englisch
                              % Für korrekt sortierte Literatureinträge, noch preambel.tex anpassen

% Titel der Arbeit auf Deutsch
\newcommand{\hsmatitelde}{Konzepterstellung zur Optimierung der Energieflüsse innerhalb einer Energieverbundinsel mit teilautarker Ladeinfrastruktur für Elektrofahrzeuge}

% Titel der Arbeit auf Englisch
\newcommand{\hsmatitelen}{Concept developement to optimize the energyflows within an island mode energy network with partially self-sufficient charging infrastructure for electric vehicles}

% Weitere Informationen zur Arbeit
\newcommand{\hsmaort}{Mannheim}    % Ort
\newcommand{\hsmaautorvname}{Matthias} % Vorname(n)
\newcommand{\hsmaautornname}{Werle} % Nachname(n)
\newcommand{\hsmadatum}{18.06.2018} % Datum der Abgabe
\newcommand{\hsmajahr}{2018} % Jahr der Abgabe
\newcommand{\hsmafirma}{} % Firma bei der die Arbeit durchgeführt wurde
\newcommand{\hsmabetreuer}{Prof. Dipl.-Ing. T. Hansemann, Hochschule Mannheim} % Betreuer an der Hochschule
\newcommand{\hsmazweitkorrektor}{Prof. Dr.-Ing. W. Götzmann, Hochschule Mannheim} % Betreuer im Unternehmen oder Zweitkorrektor
\newcommand{\hsmafakultaet}{E} % I für Informatik
\newcommand{\hsmastudiengang}{EB} % IB IMB UIB IM MTB

% Zustimmung zur Veröffentlichung
\setboolean{hsmapublizieren}{true}   % Einer Veröffentlichung wird zugestimmt
\setboolean{hsmasperrvermerk}{false} % Die Arbeit hat keinen Sperrvermerk

% -------------------------------------------------------
% Abstract

% Kurze (maximal halbseitige) Beschreibung, worum es in der Arbeit geht auf Deutsch

\newcommand{\hsmaabstractde}{Im Zuge der Energiewende wandelt sich der grundlegende Aufbau moderner elektrischer Versorgungsnetze. Die Integration vieler kleiner, örtlich verteilter, gegebenenfalls nicht regelbarer Erzeugungsanlagen erfordert gegenüber der Nutzung weniger, konventioneller, thermischer Großkraftwerke neue Konzepte für das regionale und überregionale Energiemanagement. Ziel der Studie ist die Optimierung von Energieflüssen innerhalb einer Energieverbundinsel mit teilautarker Ladeinfrastruktur. Hierfür werden die folgenden Aufgaben bearbeitet.\\

Auf Basis einer Anwendungsfallanalyse ist ein Konzept für die Energieverbundinsel und deren Energiemanagementsystem mit Orientierung am Smart Grid Architecture Model (SGAM) entworfen worden. Das Konzept gliedert sich in eine Funktions-, Informations- und Komponentenebene. \\

Für die Optimierung der Energieflüsse, sind Rahmenbedingungen verschiedener Beispielszenarien definiert, an denen das erstellte Energiemanagementkonzept getestet wurde. Für die Berechnung der Energieflüsse der einzelnen Szenarien wurde ein Simulationstool in Matlab programmiert, welches für zukünftige Untersuchungen von Energieflüssen unter anderen Bedingungen genutzt werden kann. \\

Aus den Simulationsergebnissen sind Handlungsempfehlungen zur Dimensionierung und Betriebsweise der Energieverbundinsel abgeleitet worden. Ein Ausblick zeigt die Möglichkeiten zukünftiger Entwicklungen auf.
}

% Kurze (maximal halbseitige) Beschreibung, worum es in der Arbeit geht auf Englisch

\newcommand{\hsmaabstracten}{... englische Übersetzung... }
