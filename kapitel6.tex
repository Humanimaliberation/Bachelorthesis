\chapter{Fazit}
\label{Kap6}
% Ziel und Konzept Energieverbundinsel
	Das Ziel der Studie ist es, die Energieflüsse innerhalb der geplanten Energieverbundinsel zu optimieren. Im Folgenden wird eine Zusammenfassung der Studienergebnisse und eine kritische Schlussbemerkung zur verwendeten Methodik gegeben. \\

% Konzept Energieverbundinsel
    Das Konzept der Energieverbundinsel mit Ladeinfrastruktur, PV-Anlage und Pufferbatterie wurde in die Komponenten-, Informations- und Funktionsebene des Smart Grid Architecture Model eingeordnet, um das Projekt möglichst gut in Teilbereiche zu gliedern und eine bessere Übersicht für zukünftige Entwicklungen zu schaffen. Es wurde ein möglichst modulares Modell von einer steuerbaren Ladeinfrastruktur mit PV-Anlage und Pufferbatterie erstellt, das leicht erweitert werden kann. \\

% Konzept Energiemanagement und Simlation(stool)
	Das erstellte Energiemanagementkonzept zur Optimierung der Energieflüsse ist ein ausbaubares Grundgerüst, kann funktional erweitert und auf zusätzliche Betriebsmittel ausgeweitet werden. Es wurde für verschiedene Beispielszenarien in einer programmierten Simulationsumgebung getestet, welche für weitere Untersuchungen genutzt und verbessert werden kann. Aus den Simulationsergebnissen wurden Handlungsempfehlungen für die Auswahl der Komponenten der Ladeinfrastruktur und deren Dimensionierung und Betriebsweise abgeleitet. \\
    
% Handlungsempfehlung und Ausblick    
    Es wird von einer Auslegung der Ladeinfrastruktur für Schnellladungen bis 50~kW abgeraten, da in der Regel längere Ladezeiten während der Arbeit oder zu Hause möglich sind. Für die Betriebsweise empfiehlt es sich, den Fokus auf die Entwicklung intelligenter Ladesequenzen zu legen, statt in teure Batteriespeicher zu investieren. Allgemein rät es sich, das gesamte Hochschulnetz in die Energieverbundinsel mit einzubeziehen und zusätzliche oder bestehende Betriebsmittel ins Energiemanagementsystem (EMS) zu integrieren. Als mögliche Stoßrichtung für Entwicklungen auf lange Sicht, wurde im Ausblick die Möglichkeit eines Ausbaus der Energieverbundinsel zum virtuellen Kraftwerk skizziert, wodurch Regelleistung für das Versorgungsnetz bereit gestellt werden könnte. Grundsätzlich wird geraten vor der Implementation eines EMS, das Potential verschiedener zentraler und dezentraler Kommunikationsansätze zu vergleichen. \\

% Kritische Betrachtung
	Kritisch anzumerken ist zum Einen, dass sich die Abgrenzung der Themenstellung als ungünstig umfassend erwies, sodass der Zeitaufwand auf 4 Monate erweitert werden musste und an einigen Stellen dennoch unerwünschte Abstriche gemacht werden mussten, wodurch die gewählte Aufgabenstellung nicht im vollen Umfang gelöst wurde. Zum Anderen hat sich gezeigt, dass sich die Kosten- und Energieeffizienz besser als Bewertungskriterium für verlustbehaftete Speichermöglichkeiten wie der Pufferbatterie eignen als der alleinige Vergleich erreichbarer Autarkiegrade. \\
    
% Kritische Betrachtung: Abgrenzung und Priorisierung
	Der Arbeitsaufwand zum Programmieren der halbautomatisierten Analyse regionaler Globalstrahlungsdaten zur Erstellung von PV-Ertragsprofilen wurde unterschätzt und ist unverhältnismäßig hoch. Die einmalige Auswahl jeweils eines Tages mit maximaler, durchschnittlicher und minimaler Globalstrahlung konnte mithilfe eines herkömmlichen Tabellenkalkulators schneller durchgeführt werden. Zudem hätte in der Simulationsumgebung aufgrund der untersuchten Nicht-Rentabilität der gewählten Pufferbatterie die Implementation flexibler Ladesequenzen höher priorisiert werden sollen als die Implementation der Pufferbatterie. \\
    
    Mit der Studie konnte die Zielsetzung trotz umfassender Arbeitsplanung erreicht werden. Darüber hinaus wurden Kenntnisse zu Smart Grid Architekturen, Energiemanagementkonzepten, dem Aufbau einer Ladeinfrastruktur, Programmierkenntnisse in Matlab und Methoden wissenschaftlichen Arbeitens erworben beziehungsweise ausgeweitet.



    
    
    %Es bestimmt die Nennleistung von Ladepunkten maximal oder kontinuierlich über einen Anschlusszeitraum und die Nennleistung der Pufferbatterie so, dass möglichst viel PV-Ertragsüberschüsse gepuffert und bei Bedarf bereit gestellt werden. Im Falle, dass die Gesamtsumme aller Nennleistungen von Geräten an einem Netzknoten die zulässigen Übertragungsleistung zum nächsten verbundenen Netzknoten überschreitet, werden stufenweise alle Geräte eines Netzknotens mit identischer Priorität gleichmäßig gedrosselt, um möglichst viel der unzulässigen Nennleistung auszugleichen. \\

% Simulation, Auswertung, Handlungsempfehlungen Komponenten und Energiemanagement
    